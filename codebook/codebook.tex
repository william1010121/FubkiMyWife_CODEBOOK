\documentclass[a4paper,10pt,twocolumn,oneside]{article}
\setlength{\columnsep}{10pt}                                                                    %兩欄模式的間距
\setlength{\columnseprule}{0pt}                                                                %兩欄模式間格線粗細

\usepackage{amsthm}								%定義,例題
\usepackage{amssymb}
%\usepackage[margin=2cm]{geometry}
\usepackage{fontspec}								%設定字體
\usepackage{color}
\usepackage[x11names]{xcolor}
\usepackage{listings}								%顯示code用的
%\usepackage[Glenn]{fncychap}						%排版,頁面模板
\usepackage{fancyhdr}								%設定頁首頁尾
\usepackage{graphicx}								%Graphic
\usepackage{enumerate}
\usepackage{multicol}
\usepackage{titlesec}
\usepackage{amsmath}
\usepackage[CheckSingle, CJKmath]{xeCJK}
\usepackage{savetrees}
\usepackage{array}
\usepackage{xparse}
% \usepackage{CJKulem}

%\usepackage[T1]{fontenc}
\usepackage{amsmath, courier, listings, fancyhdr, graphicx}
\topmargin=0pt
\headsep=5pt
\textheight=780pt
\footskip=0pt
\voffset=-40pt
\textwidth=545pt
\marginparsep=0pt
\marginparwidth=0pt
\marginparpush=0pt
\oddsidemargin=0pt
\evensidemargin=0pt
\hoffset=-42pt

\titlespacing\section{0pt}{-2pt plus 0pt minus 2pt}{-1pt plus 0pt minus 2pt}
\titlespacing\subsection{0pt}{-2pt plus 0pt minus 2pt}{-1pt plus 0pt minus 2pt}
\titlespacing\subsubsection{0pt}{-2pt plus 0pt minus 2pt}{-1pt plus 0pt minus 2pt}


%\renewcommand\listfigurename{圖目錄}
%\renewcommand\listtablename{表目錄} 

%%%%%%%%%%%%%%%%%%%%%%%%%%%%%

%\setmainfont{Ubuntu}				%主要字型
%\setmonofont{Ubuntu Mono}
\XeTeXlinebreaklocale "zh"						%中文自動換行
\XeTeXlinebreakskip = 0pt plus 1pt				%設定段落之間的距離
\setcounter{secnumdepth}{3}						%目錄顯示第三層

%%%%%%%%%%%%%%%%%%%%%%%%%%%%%
\newcommand\digitstyle{\color{DarkOrchid3}}
\makeatletter
\lst@CCPutMacro\lst@ProcessOther {"2D}{\lst@ttfamily{-{}}{-{}}}
\@empty\z@\@empty

\newtoks\BBQube@token
\newcount\BBQube@length
\def\BBQube@ResetToken{\BBQube@token{}\BBQube@length\z@}
\def\BBQube@Append#1{\advance\BBQube@length\@ne
  \BBQube@token=\expandafter{\the\BBQube@token#1}}

\def\BBQube@ProcessChar#1{%
  \ifnum\lst@mode=\lst@Pmode%
    \ifnum 9<1#1%
      \expandafter\BBQube@Append{\begingroup\digitstyle #1 \endgroup}%
    \else%
      \expandafter\BBQube@Append{#1}%
    \fi%
  \else%
    \expandafter\BBQube@Append{#1}%
  \fi%
}
\def\BBQube@ProcessStringInner#1#2\BBQube@nil{%
  \expandafter\BBQube@ProcessChar{#1}%
  \if\relax\detokenize{#2}\relax%
  \else%
    \expandafter\BBQube@ProcessStringInner#2\BBQube@nil%
  \fi%
}

\def\BBQube@ProcessString#1{\expandafter\BBQube@ProcessStringInner#1\BBQube@nil}

\lst@AddToHook{OutputOther}{%
\BBQube@ResetToken%
\expandafter\BBQube@ProcessString{\the\lst@token}%
\lst@token=\expandafter{\the\BBQube@token}%
}
\makeatother
\lstset{											% Code顯示
language=C++,										% the language of the code
basicstyle=\footnotesize\ttfamily, 						% the size of the fonts that are used for the code
%numbers=left,										% where to put the line-numbers
numberstyle=\footnotesize,						% the size of the fonts that are used for the line-numbers
stepnumber=1,										% the step between two line-numbers. If it's 1, each line  will be numbered
numbersep=5pt,										% how far the line-numbers are from the code
backgroundcolor=\color{white},					% choose the background color. You must add \usepackage{color}
showspaces=false,									% show spaces adding particular underscores
showstringspaces=false,							% underline spaces within strings
showtabs=false,									% show tabs within strings adding particular underscores
frame=false,											% adds a frame around the code
tabsize=2,											% sets default tabsize to 2 spaces
captionpos=b,										% sets the caption-position to bottom
breaklines=true,									% sets automatic line breaking
breakatwhitespace=false,							% sets if automatic breaks should only happen at whitespace
escapeinside={\%*}{*)},							% if you want to add a comment within your code
morekeywords={constexpr},									% if you want to add more keywords to the set
keywordstyle=\bfseries\color{Blue1},
commentstyle=\itshape\color{Red4},
stringstyle=\itshape\color{Green4},
}

%%%%%%%%%%%%%%%%%%%%%%%%%%%%%

\ExplSyntaxOn
\NewDocumentCommand{\captureshell}{som}
 {
  \sdaau_captureshell:Ne \l__sdaau_captureshell_out_tl { #3 }
  \IfBooleanT { #1 }
   {% we may need to stringify the result
    \tl_set:Nx \l__sdaau_captureshell_out_tl
     { \tl_to_str:N \l__sdaau_captureshell_out_tl }
   }
  \IfNoValueTF { #2 }
   {
    \tl_use:N \l__sdaau_captureshell_out_tl
   }
   {
    \tl_set_eq:NN #2 \l__sdaau_captureshell_out_tl
   }
 }

\tl_new:N \l__sdaau_captureshell_out_tl

\cs_new_protected:Nn \sdaau_captureshell:Nn
 {
  \sys_get_shell:nnN { #2 } { } #1
  \tl_trim_spaces:N #1 % remove leading and trailing spaces
 }
\cs_generate_variant:Nn \sdaau_captureshell:Nn { Ne }
\ExplSyntaxOff

\begin{document}
\pagestyle{fancy}
\fancyfoot{}
%\fancyfoot[R]{\includegraphics[width=20pt]{ironwood.jpg}}
\fancyhead[L]{National Taiwan University 8BQube}
\fancyhead[R]{\thepage}
\renewcommand{\headrulewidth}{0.4pt}
\renewcommand{\contentsname}{Contents} 
\newcommand{\inputcode}[2]{
    \subsection[#1]{#1 \footnotesize{[\texttt{\captureshell{cpp #2 -dD -P -fpreprocessed | tr -d '[:space:]' | md5sum | cut -c-6}}]}}
    \lstinputlisting{#2}
}

\textbf{
\scriptsize
\begin{multicols}{2}
  \tableofcontents
\end{multicols}
}
%%%%%%%%%%%%%%%%%%%%%%%%%%%%%

%\newpage

\footnotesize
\section{Basic}
%\subsection{Shell script}
%\lstinputlisting{1_Basic/Shell_script.cpp}
%\subsection{Default code}
%\lstinputlisting{1_Basic/Default_code.cpp}
\subsection{vimrc}
\lstinputlisting{1_Basic/vimrc.cpp}
\inputcode{readchar}{1_Basic/readchar.cpp}
\inputcode{BigIntIO}{1_Basic/BigIntIO.cpp}
\inputcode{Black Magic}{1_Basic/black_magic.cpp}
\inputcode{Pragma Optimization}{1_Basic/Pragma.cpp}
\inputcode{Bitset}{1_Basic/bitset.cpp}
% \subsection{Texas hold'em}
% \lstinputlisting{1_Basic/Texas_holdem.cpp}


\section{Graph}
\inputcode{BCC Vertex*}{2_Graph/BCC_Vertex.cpp} % test by Library Checker Biconnected Components, block-cut tree test by CF 102512 A
\inputcode{Bridge*}{2_Graph/Bridge.cpp} % test by Library Checker Two-Edge-Connected Components
\inputcode{SCC*}{2_Graph/SCC.cpp} % SCC test by Library Checker Strongly Connected Components
\inputcode{2SAT*}{2_Graph/2SAT.cpp} % test by ARC069 F
\inputcode{MinimumMeanCycle*}{2_Graph/MinimumMeanCycle.cpp} % test by TIOJ 1934
\inputcode{Virtual Tree*}{2_Graph/Virtual_Tree.cpp} % test by luogu P2495
\inputcode{Maximum Clique Dyn*}{2_Graph/Maximum_Clique_Dyn.cpp} % test by TIOJ 1978, CF 101221 I (World Finals' problem)
\inputcode{Minimum Steiner Tree*}{2_Graph/MinimumSteinerTree.cpp} % test by luogu P6192
\inputcode{Dominator Tree*}{2_Graph/Dominator_Tree.cpp} % test by CF 100513 L
%\inputcode{Minimum Arborescence*}{2_Graph/Minimum_Arborescence_fast.cpp} % test by luogu P4716 
%\inputcode{Vizing's theorem*}{2_Graph/Vizing.cpp} % test by CF 101933 G
\inputcode{Four Cycle}{2_Graph/FourCircleCount.cpp}
\inputcode{Minimum Clique Cover*}{2_Graph/Minimum_Clique_Cover.cpp} % test by TIOJ 1472
\inputcode{NumberofMaximalClique*}{2_Graph/NumberofMaximalClique.cpp} % test by POJ 2989


\section{Data Structure}
\subsection{Discrete Trick}
\lstinputlisting{3_Data_Structure/discrete_trick.cpp}
\inputcode{BIT kth*}{3_Data_Structure/BIT_kth.cpp} % test by CSES 1076
\inputcode{Interval Container*}{3_Data_Structure/IntervalContainer.cpp} % test by kactl stress test
\inputcode{Leftist Tree}{3_Data_Structure/Leftist_Tree.cpp}
\inputcode{Heavy light Decomposition*}{3_Data_Structure/Heavy_light_Decomposition.cpp} % test by CSES Path Queries II
\inputcode{Centroid Decomposition*}{3_Data_Structure/Centroid_Decomposition.cpp} % test by TIOJ 1171
% \inputcode{Smart Pointer}{3_Data_Structure/Smart_Pointer.cpp}
\inputcode{LiChaoST*}{3_Data_Structure/LiChaoST.cpp} % test by Library Checker Line Add Get Min
\inputcode{Link cut tree*}{3_Data_Structure/link_cut_tree.cpp} % test by luogu P3690
\inputcode{KDTree}{3_Data_Structure/KDTree.cpp}
\inputcode{Treap}{3_Data_Structure/Treap.cpp}
% \inputcode{Range Chmin Chmax Add Range Sum*}{3_Data_Structure/Range_Chmin_Chmax_Add_Range_Sum.cpp} % test by Lib-Checker Range Chmin Chmax Add Range Sum


\section{Flow/Matching}
\inputcode{Dinic}{4_Flow_Matching/Dinic.cpp}
\inputcode{Bipartite Matching*}{4_Flow_Matching/Bipartite_Matching.cpp} % test by Lib-Checker Matching on Bipartite Graph
\inputcode{Kuhn Munkres*}{4_Flow_Matching/Kuhn_Munkres.cpp} % test by CF 101239 C (World Finals' problem)
\inputcode{MincostMaxflow*}{4_Flow_Matching/MincostMaxflow.cpp} % test by luogu 3381
\inputcode{Maximum Simple Graph Matching*}{4_Flow_Matching/Maximum_Simple_Graph_Matching.cpp} % test by Library Checker Matching on General Graph 
\inputcode{Maximum Weight Matching*}{4_Flow_Matching/Maximum_Weight_Matching.cpp} % test by Library Checker General Weighted Matching
\inputcode{SW-mincut}{4_Flow_Matching/SW-mincut.cpp}
\inputcode{BoundedFlow*(Dinic*)}{4_Flow_Matching/BoundedFlow.cpp} % Maximum boundedflow test by LOJ 116; Minimum boundedflow test by LOJ 117; Dinic test by NTUJ 184
\inputcode{Gomory Hu tree*}{4_Flow_Matching/Gomory_Hu_tree.cpp} % test by LOJ 2042
\inputcode{Minimum Cost Circulation*}{4_Flow_Matching/MinCostCirculation.cpp} % test by uoj 487 
\subsection{Flow Models}
\input{4_Flow_Matching/Model.tex}
\subsection{matching}
\begin{itemize}
	\item 最大匹配 $+$ 最小邊覆蓋 $= V$
	\item 最大獨立集 $+$ 最小點覆蓋 $= V$
	\item 最大匹配 $=$ 最小點覆蓋
	\item 最小路徑覆蓋數 $= V -$ 最大匹配數
\end{itemize}
% \inputcode{isap}{4_Flow_Matching/isap.cpp}


\section{String}
\inputcode{KMP}{5_String/KMP.cpp}
\inputcode{Z-value*}{5_String/Z-value.cpp} % test by Lib-Checker Z Algorithm
\inputcode{Manacher*}{5_String/Manacher.cpp} % test by Lib-Checker Enumerate Palindromes
%\inputcode{Suffix Array}{5_String/Suffix_Array.cpp}
\inputcode{SAIS*}{5_String/SAIS-C++20.cpp} % test by CF 104901 H, Lib-Checker Suffix Array
\inputcode{Aho-Corasick Automatan*}{5_String/Aho-Corasick_Automatan.cpp} % test by CF 102511 G (World Finals' problem)
\inputcode{Smallest Rotation}{5_String/Smallest_Rotation.cpp}
\inputcode{De Bruijn sequence*}{5_String/De_Bruijn_sequence.cpp} % test by CF 102001 C
\inputcode{Extended SAM*}{5_String/exSAM.cpp} % test by CF 616 C
\inputcode{PalTree*}{5_String/PalTree.cpp} % test by APIO 2014 palindrome
\inputcode{Main Lorentz}{5_String/MainLorentz.cpp}


\section{Math}
\inputcode{ax+by=gcd(only exgcd *)}{6_Math/ax+by=gcd.cpp} % exgcd test by NTUJ 110
\inputcode{Floor and Ceil}{6_Math/floor_ceil.cpp}
\inputcode{Floor Enumeration}{6_Math/floor_enumeration.cpp}
\inputcode{Mod Min}{6_Math/ModMin.cpp}
\inputcode{Linear Mod Inverse}{6_Math/Mod_Inverse.cpp}
\inputcode{Linear Filter Mu}{6_Math/getMu.cpp}
\inputcode{Gaussian integer gcd}{6_Math/Gaussian_gcd.cpp}
\inputcode{GaussElimination}{6_Math/Gaussian_Eliminatin.cpp}
\inputcode{floor sum*}{6_Math/floor_sum.cpp} % test by AtCoder Library Practice Contest C, CF 100920 J
\inputcode{Miller Rabin*}{6_Math/Miller_Rabin.cpp} % test by NTUJ 1237
\inputcode{Big number}{6_Math/Big_number.cpp}
\inputcode{Fraction}{6_Math/Fraction.cpp}
\inputcode{Simultaneous Equations}{6_Math/Simultaneous_Equations.cpp}
\inputcode{Pollard Rho*}{6_Math/Pollard_Rho.cpp} % test by Lib-Checker Factorize
\inputcode{Simplex Algorithm}{6_Math/Simplex_Algorithm.cpp}
\subsubsection{Construction}
\input{6_Math/SimplexConstruction.tex}
%\inputcode{Schreier-Sims Algorithm*}{6_Math/SchreierSims.cpp} % test by XVI Opencup GP of Ekaterinburg H
\inputcode{chineseRemainder}{6_Math/chineseRemainder.cpp}
\inputcode{Factorial without prime factor*}{6_Math/fac_no_p.cpp} % test by luogu P4720
%\inputcode{QuadraticResidue*}{6_Math/QuadraticResidue.cpp} % test by Lib-Checker Sqrt Mod
\inputcode{PiCount*}{6_Math/PiCount.cpp} % test by luogu P7884
\inputcode{Discrete Log*}{6_Math/DiscreteLog.cpp} % test by Lib-Checker Discrete Logarithm
\inputcode{Berlekamp Massey}{6_Math/Berlekamp-Massey.cpp}
\subsection{Primes}
\lstinputlisting{6_Math/Primes.cpp}
%\subsection{Theorem}
%\input{6_Math/Theorem.tex}
\subsection{Estimation}
\input{6_Math/Estimation.tex}
%\subsection{Euclidean Algorithms}
%\input{6_Math/Euclidean.tex}
\subsection{General Purpose Numbers}
\input{6_Math/numbers.tex}
\subsection{Tips for Generating Functions}
\input{6_Math/Generating_function.tex}

\section{Polynomial}
\inputcode{Fast Fourier Transform}{7_Polynomial/Fast_Fourier_Transform.cpp}
\input{7_Polynomial/Number_Theory_Transform_Prime}
\inputcode{Number Theory Transform*}{7_Polynomial/Number_Theory_Transform.cpp} % test by Lib-Checker Convolution
\inputcode{Fast Walsh Transform*}{7_Polynomial/Fast_Walsh_Transform.cpp} % test by luogu P6097
\inputcode{Polynomial Operation}{7_Polynomial/Polynomial_Operation.cpp}
\inputcode{Value Polynomial}{7_Polynomial/Value_Poly.cpp}
\subsection{Newton's Method}
\input{7_Polynomial/Newton.tex}


\section{Geometry}
\inputcode{Basic}{8_Geometry/_basic.cpp}
\inputcode{KD Tree}{8_Geometry/KDTree.cpp}
%\inputcodewithtext{Delaunay Triangulation}{8_Geometry/Triangulation.cpp}{Fast Delaunay triangulation assuming no duplicates and not all points collinear (in latter case, result will be empty). Should work for doubles as well, though there may be precision issues in 'circ'.  Returns triangles in ccw order. Each circumcircle will contain none of the input points. If coordinates are ints at most $B$ then \texttt{T} should be large enough to support ints on the order of $B^4$. We don't need \texttt{double} in \texttt{Point} if the coordinates are integers.}
%\inputcodewithtext{Voronoi Diagram}{8_Geometry/Triangulation_Vonoroi_short.cpp}{Vertices in Voronoi Diagram are circumcenters of triangles in the Delaunay Triangulation.}
\inputcode{Sector Area}{8_Geometry/sectorArea.cpp}
\inputcode{Half Plane Intersection}{8_Geometry/halfPlaneIntersect.cpp}
\inputcode{Rotating Sweep Line}{8_Geometry/rotatingSweepLine.cpp}
\inputcode{Triangle Center}{8_Geometry/triangleCenter.cpp}
\inputcode{Polygon Center}{8_Geometry/center.cpp}
\inputcode{Maximum Triangle}{8_Geometry/maxTriangleOfConvex.cpp}
\inputcode{Point in Polygon}{8_Geometry/pointInPolygon.cpp}
\inputcode{Circle}{8_Geometry/circle.cpp}
\inputcode{Tangent of Circles and Points to Circle}{8_Geometry/CircleTangent.cpp}
\inputcode{Area of Union of Circles}{8_Geometry/circleOrArea.cpp}
\inputcode{Minimun Distance of 2 Polygons}{8_Geometry/minDistOfTwoConvex.cpp}
\inputcode{2D Convex Hull}{8_Geometry/convexHull.cpp}
\inputcode{3D Convex Hull}{8_Geometry/convex3D.cpp}
\inputcode{Minimum Enclosing Circle}{8_Geometry/MinimumEnclosingCircle.cpp}
\inputcode{Closest Pair}{8_Geometry/ClosestPair.cpp}

\section{Else}
\inputcode{Cyclic Ternary Search*}{9_Else/cyc_tsearch.cpp} % test by local brute force
\inputcode{Mo's Algorithm(With modification)}{9_Else/Mos_Algorithm_With_modification.cpp}
\inputcode{Mo's Algorithm On Tree}{9_Else/Mos_Algorithm_On_Tree.cpp}
\subsection{Additional Mo's Algorithm Trick}
\input{9_Else/Mos_Algorithm.tex}
\inputcode{Hilbert Curve}{9_Else/HilbertCurve.cpp}
\inputcode{DynamicConvexTrick*}{9_Else/DynamicConvexTrick.cpp} % test by TIOJ 1921
%\inputcode{cyclicLCS}{9_Else/cyclicLCS.cpp}
\inputcode{All LCS*}{9_Else/All_LCS.cpp} % test by Library Checker prefix-substring LCS
%\inputcode{DLX*}{9_Else/DLX.cpp} % test by TIOJ 1333, 1381
%\subsection{Matroid Intersection}
%\input{9_Else/Matroid.tex}
\inputcode{AdaptiveSimpson*}{9_Else/AdaptiveSimpson.cpp} % test by CF 101193J, 100553D
\inputcode{Simulated Annealing}{9_Else/simulated_annealing.cpp}
\inputcode{Tree Hash*}{9_Else/tree_hash.cpp} % test by CSES Tree Isomorphism I
\inputcode{Binary Search On Fraction}{9_Else/BinarySearchOnFraction.cpp}
%\inputcode{Min Plus Convolution*}{9_Else/min_plus_convolution.cpp} % test by Library Checker Min Plus Convolution (Convex and Arbitrary)
\inputcode{Bitset LCS}{9_Else/BitsetLCS.cpp}
\inputcode{N Queens Problem}{9_Else/NQueens.cpp}
\section{Python}
\subsection{Misc}
\lstinputlisting[language=Python]{11_Python/misc.py}


\end{document}
